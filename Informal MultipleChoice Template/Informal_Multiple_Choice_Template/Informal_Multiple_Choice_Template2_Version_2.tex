%!TEX encoding = UTF-8 Unicode
% Created by Alejandro Gonzalez Recuenco, 2017
% Do not share outside of BFITS
% e-mail alejandrogonzalezrecuenco@gmail.com

\documentclass[12pt, noquestionbreak, flushbottom, customfontpath=../Fonts/]{../skrexam}  %Draft option also removes the pictures

\newtotcounter{totalpages}
\newtotcounter{totalsections}
\usepackage{tcolorbox}
\tcbuselibrary{hooks}
\usepackage{physics}

\graphicspath{{../images/}{images/}}

%  Showcasing commands for TexExamRandomizer
%! TexExamRandomizer = {"noutput":3}
%! TexExamRandomizer = {"randominfo": {"largeseed":100000000}}
%! DISABLED TexExamRandomizer = {"table":"../Tables/TestClass.csv"}
%! DISABLED TexExamRandomizer = {"randominfo": {"switchnumber":["even", "odd"]}}
%! DISABLED TexExamRandomizer = {"extrainfo":{"class":"Class"}}
%! DISABLED TexExamRandomizer =  {"extrainfo":{"rollnumber":"Roll Number","nickname":"Nickname"}, "seed":69504}



\newcommand{\rseed}{1516016468} % number between 1 and 99999999
\newcommand{\largeseed}{26369981}

\pgfmathsetseed{\largeseed} % Setting the random number generation



\newcommand{\myversion}{2} % this line is very (DO NOT DELETE)

\newcommand{\studentclass}{M 6}



\newcommand*\SchoolName{\textbf{\large Suankularb wittayalai Rangsit School}}
\newcommand*\SchoolNameTH{\textbf{\small โรงเรียนสวนกุหลาบวิทยาลัย รังสิต}}


%<Header & Footer>%
\pagestyle{headandfoot}
\firstpagefooter{Exam Version: \myversion}{\studentclass}{\thepage}
\runningfooter{Exam Version: \myversion}{\studentclass}{\thepage}


\runningfootrule
\firstpagefootrule

\begin{document}

\global\edef\nmultiplechoice{\totalvalue{nummultiplechoice}}
\global\edef\ntotalquestion{\totalvalue{questionnumber}}
\global\edef\nshortanswer{\totalvalue{numshortanswer}}


\begin{coverpages}


\begin{center}
\SchoolName~---~\SchoolNameTH

\vspace{1cm}

%%% TITLE OF EXAM %%%
\textsc{\textbf{\Huge TITLE OF EXAM HERE}}

\vspace{1cm}

\begin{tcolorbox}[title ={\large Class~\studentclass\quad\quad Exam Version: \myversion}, center title, colback = white]

Name: \hrulefill\ Nickname:\hrulefill

\vspace{0.5cm}

\raisebox{-0.5\height}{\includegraphics[height = 3cm]{SKR.jpg}}
\hfill $\begin{aligned}\text{Roll Number: \rule{2cm}{0.5pt}}\\ \text{ID: \rule{2cm}{0.5pt}}\end{aligned}$ \hfill 
\raisebox{-0.5\height}{\includegraphics[height = 3cm]{BFITS}}


\tcblower

\begin{itemize}
	\item This quiz contains \total{totalpages} pages. 
%	\item In those pages there are \total{totalsections} parts to the exam
\end{itemize}
\end{tcolorbox}

\vspace{1cm}


\section*{Choose the correct answer} %Short Answer	

\fullanswersheet[10]{\nmultiplechoice}


\vspace{\stretch{1.5}}

\end{center}

\cleardoublepage

\end{coverpages}




%%% QUESTIONS GO HERE
%% \begin{choices}(2) means ``two columns''
%% \begin{choices}(1) means ``one column''


\begin{questions}

	\question Given a continuous function at every point in the real line, $f(x)$. Which of these statements is true. 
	\begin{choices}(1)
		\choice The function $f(x)$ jumps the $y$ axis.
		\choice The function $f(x)$ must cross the $x$ axis at least once.
		\choice The function $f(x)$ never crosses the $x$ axis.
		\choice\CorrectChoice For all values $a$ on the real line, $ \lim_{x \to a}f(x)  = 0$
	\end{choices}
	

	
\question 
	\pgfmathtruncatemacro{\coeficient}{random(2,10)} 
	\pgfmathtruncatemacro{\exponent}{random(3, 10)} % MUST BE BIGEERN THAN 2 TO WORK
	\pgfmathtruncatemacro{\resultexponent}{\exponent-1}
	\pgfmathtruncatemacro{\resultcoeficient}{\coeficient*\exponent}
	What is the value of $\dv{\qty(\coeficient x^{\exponent})}{x}$
	\begin{choices}
		\choice\CorrectChoice $\dv{\qty(\coeficient x^{\exponent})}{x} = \resultcoeficient x^{\resultexponent}$
		\choice $\dv{\qty(\coeficient x^{\exponent})}{x} = \coeficient x^{\exponent}$
		\choice $\dv{\qty(\coeficient x^{\exponent})}{x} = \coeficient x^{\resultexponent}$
		\choice $\dv{\qty(\coeficient x^{\exponent})}{x} = \resultcoeficient x$
%		\choice $\dv{\qty(\coeficient x^{\exponent})}{x} = \resultcoeficient x^{\exponent}$
	\end{choices}

\question What is the definition of the derivative of $f(x)$
 	\begin{choices}(2)
	\choice $\displaystyle f^\prime(x) = \lim_{h \to 0} \frac{f(x+h)-f(x)}{x}$	
	\choice $\displaystyle f^\prime(x) = \lim_{h \to 0} \frac{f(x)}{x}$
	\choice $\displaystyle f^\prime(x) = \lim_{h \to 0} \frac{f(x) - f(h)}{h}$
 	\choice\CorrectChoice $\displaystyle f^\prime(x) = \lim_{h \to 0} \frac{f(x+h)-f(x)}{h}$
 	\end{choices}

	\question According to the product rule, when I have two functions, $f(x)$ and $g(x)$, what is the derivative of $h(x) = f(x) g(x)$?
	\begin{choices}
		\choice $h'(x) = f(x) g '(x)$.
		\choice $h'(x) = f'(x) g '(x)$.
		\choice $h'(x) = f'(x) g(x)$.
		\choice\CorrectChoice $h'(x) = f'(x) g(x)+ f(x) g '(x)$.
	\end{choices}


	\question What is the derivative of  $f(x) = (x + 2)(2x- 1)$,
	\begin{choices}
		\choice $f'(x) = 2$
		\choice\CorrectChoice $f'(x) = 4 x + 3$
		\choice $f'(x) = 2x - 1$
		\choice $f'(x) = 2x + 4$
	\end{choices}




	\question According to the product rule, when I have two functions, $f(x)$ and $g(x)$, what is the derivative of $h(x) = f(x) g(x)$:
	\begin{choices}
		\choice $h'(x) = f'(x) g '(x)$
		\choice\CorrectChoice $h'(x) = f'(x) g(x)+ f(x) g '(x)$
		\choice $h'(x) = f'(x) g(x)$
		\choice $h'(x) = f(x) g '(x)$
	\end{choices}

	\question What does the first derivative being 0 tells us?
	\begin{choices}(1)
		\choice When the first derivative is $0$ at one point, $f'(a) = 0$, that means that the point is always an inflection of the graph.
		\choice\CorrectChoice When the first derivative is $0$ at one point, $f'(a) = 0$, that means that the slope of the graph is $0$ at that point
		\choice When the first derivative is $0$ at one point, $f'(a) = 0$, that means that the graph of the function crosses the $x$ axes on that point
		\choice When the first derivative is $0$ at one point, $f'(a) = 0$, that means that the point is always a local maximum or a local minimum of the graph.
	\end{choices}

	\question What does the second derivative being 0 tells us?
	\begin{choices}(1)
		\choice When the first derivative is $0$ at one point, $f^{\prime \prime}(a) = 0$, that means that the point is always a local maximum or a local minimum of the graph.
		\choice When the first derivative is $0$ at one point, $f^{\prime \prime}(a) = 0$, that means that the slope of the graph is $0$ at that point.
		\choice When the first derivative is $0$ at one point, $f^{\prime \prime}(a) = 0$, that means that the graph of the function crosses the $x$ axes on that point.
		\choice\CorrectChoice When the first derivative is $0$ at one point, $f^{\prime \prime}(a) = 0$, that means that the point is always an inflection of the graph.
	\end{choices}

\end{questions}






\[
\underset{% 
%	\text{End of \thesection}%
	}{%
	\text{\underline{\hspace{\textwidth}}}%
}
\]
\setcounter{totalpages}{\value{page}}
\setcounter{totalsections}{\value{section}}

\end{document}
